% Options for packages loaded elsewhere
\PassOptionsToPackage{unicode}{hyperref}
\PassOptionsToPackage{hyphens}{url}
%
\documentclass[
]{article}
\usepackage{amsmath,amssymb}
\usepackage{lmodern}
\usepackage{iftex}
\ifPDFTeX
  \usepackage[T1]{fontenc}
  \usepackage[utf8]{inputenc}
  \usepackage{textcomp} % provide euro and other symbols
\else % if luatex or xetex
  \usepackage{unicode-math}
  \defaultfontfeatures{Scale=MatchLowercase}
  \defaultfontfeatures[\rmfamily]{Ligatures=TeX,Scale=1}
\fi
% Use upquote if available, for straight quotes in verbatim environments
\IfFileExists{upquote.sty}{\usepackage{upquote}}{}
\IfFileExists{microtype.sty}{% use microtype if available
  \usepackage[]{microtype}
  \UseMicrotypeSet[protrusion]{basicmath} % disable protrusion for tt fonts
}{}
\makeatletter
\@ifundefined{KOMAClassName}{% if non-KOMA class
  \IfFileExists{parskip.sty}{%
    \usepackage{parskip}
  }{% else
    \setlength{\parindent}{0pt}
    \setlength{\parskip}{6pt plus 2pt minus 1pt}}
}{% if KOMA class
  \KOMAoptions{parskip=half}}
\makeatother
\usepackage{xcolor}
\usepackage[margin=1in]{geometry}
\usepackage{color}
\usepackage{fancyvrb}
\newcommand{\VerbBar}{|}
\newcommand{\VERB}{\Verb[commandchars=\\\{\}]}
\DefineVerbatimEnvironment{Highlighting}{Verbatim}{commandchars=\\\{\}}
% Add ',fontsize=\small' for more characters per line
\usepackage{framed}
\definecolor{shadecolor}{RGB}{248,248,248}
\newenvironment{Shaded}{\begin{snugshade}}{\end{snugshade}}
\newcommand{\AlertTok}[1]{\textcolor[rgb]{0.94,0.16,0.16}{#1}}
\newcommand{\AnnotationTok}[1]{\textcolor[rgb]{0.56,0.35,0.01}{\textbf{\textit{#1}}}}
\newcommand{\AttributeTok}[1]{\textcolor[rgb]{0.77,0.63,0.00}{#1}}
\newcommand{\BaseNTok}[1]{\textcolor[rgb]{0.00,0.00,0.81}{#1}}
\newcommand{\BuiltInTok}[1]{#1}
\newcommand{\CharTok}[1]{\textcolor[rgb]{0.31,0.60,0.02}{#1}}
\newcommand{\CommentTok}[1]{\textcolor[rgb]{0.56,0.35,0.01}{\textit{#1}}}
\newcommand{\CommentVarTok}[1]{\textcolor[rgb]{0.56,0.35,0.01}{\textbf{\textit{#1}}}}
\newcommand{\ConstantTok}[1]{\textcolor[rgb]{0.00,0.00,0.00}{#1}}
\newcommand{\ControlFlowTok}[1]{\textcolor[rgb]{0.13,0.29,0.53}{\textbf{#1}}}
\newcommand{\DataTypeTok}[1]{\textcolor[rgb]{0.13,0.29,0.53}{#1}}
\newcommand{\DecValTok}[1]{\textcolor[rgb]{0.00,0.00,0.81}{#1}}
\newcommand{\DocumentationTok}[1]{\textcolor[rgb]{0.56,0.35,0.01}{\textbf{\textit{#1}}}}
\newcommand{\ErrorTok}[1]{\textcolor[rgb]{0.64,0.00,0.00}{\textbf{#1}}}
\newcommand{\ExtensionTok}[1]{#1}
\newcommand{\FloatTok}[1]{\textcolor[rgb]{0.00,0.00,0.81}{#1}}
\newcommand{\FunctionTok}[1]{\textcolor[rgb]{0.00,0.00,0.00}{#1}}
\newcommand{\ImportTok}[1]{#1}
\newcommand{\InformationTok}[1]{\textcolor[rgb]{0.56,0.35,0.01}{\textbf{\textit{#1}}}}
\newcommand{\KeywordTok}[1]{\textcolor[rgb]{0.13,0.29,0.53}{\textbf{#1}}}
\newcommand{\NormalTok}[1]{#1}
\newcommand{\OperatorTok}[1]{\textcolor[rgb]{0.81,0.36,0.00}{\textbf{#1}}}
\newcommand{\OtherTok}[1]{\textcolor[rgb]{0.56,0.35,0.01}{#1}}
\newcommand{\PreprocessorTok}[1]{\textcolor[rgb]{0.56,0.35,0.01}{\textit{#1}}}
\newcommand{\RegionMarkerTok}[1]{#1}
\newcommand{\SpecialCharTok}[1]{\textcolor[rgb]{0.00,0.00,0.00}{#1}}
\newcommand{\SpecialStringTok}[1]{\textcolor[rgb]{0.31,0.60,0.02}{#1}}
\newcommand{\StringTok}[1]{\textcolor[rgb]{0.31,0.60,0.02}{#1}}
\newcommand{\VariableTok}[1]{\textcolor[rgb]{0.00,0.00,0.00}{#1}}
\newcommand{\VerbatimStringTok}[1]{\textcolor[rgb]{0.31,0.60,0.02}{#1}}
\newcommand{\WarningTok}[1]{\textcolor[rgb]{0.56,0.35,0.01}{\textbf{\textit{#1}}}}
\usepackage{graphicx}
\makeatletter
\def\maxwidth{\ifdim\Gin@nat@width>\linewidth\linewidth\else\Gin@nat@width\fi}
\def\maxheight{\ifdim\Gin@nat@height>\textheight\textheight\else\Gin@nat@height\fi}
\makeatother
% Scale images if necessary, so that they will not overflow the page
% margins by default, and it is still possible to overwrite the defaults
% using explicit options in \includegraphics[width, height, ...]{}
\setkeys{Gin}{width=\maxwidth,height=\maxheight,keepaspectratio}
% Set default figure placement to htbp
\makeatletter
\def\fps@figure{htbp}
\makeatother
\setlength{\emergencystretch}{3em} % prevent overfull lines
\providecommand{\tightlist}{%
  \setlength{\itemsep}{0pt}\setlength{\parskip}{0pt}}
\setcounter{secnumdepth}{-\maxdimen} % remove section numbering
\ifLuaTeX
  \usepackage{selnolig}  % disable illegal ligatures
\fi
\IfFileExists{bookmark.sty}{\usepackage{bookmark}}{\usepackage{hyperref}}
\IfFileExists{xurl.sty}{\usepackage{xurl}}{} % add URL line breaks if available
\urlstyle{same} % disable monospaced font for URLs
\hypersetup{
  pdftitle={STA302 Final Project},
  pdfauthor={Guanlin Chen},
  hidelinks,
  pdfcreator={LaTeX via pandoc}}

\title{STA302 Final Project}
\author{Guanlin Chen}
\date{10/1/2023}

\begin{document}
\maketitle

这个是我做的code,
研究影响北京市2017年交易的130平方以上的中大型房子价格的因素。
这个code里面展示了在画Part2 的流程图之前我们要先对数据测试一遍

大家按照下面的步骤把测试做一遍,variables可以自己按情况重选,model也可以按情况改进。\#\#comments要阅读

我们目标是: 1. fit出一个尽可能不violated
assumption的模型并且满足教授对model的要求 2. response 和predictors
的关系能让我们顺理成章去探究什么方面的因素影响了每平的价格。最好predictors之间会有一个在某一方面的interaction例如:出行的便捷性,从众消费心理,装修的精简程度。

Step1: Data Cleaning

\begin{verbatim}
## -- Attaching core tidyverse packages ------------------------ tidyverse 2.0.0 --
## v dplyr     1.1.2     v readr     2.1.4
## v forcats   1.0.0     v stringr   1.5.0
## v ggplot2   3.4.2     v tibble    3.2.1
## v lubridate 1.9.2     v tidyr     1.3.0
## v purrr     1.0.1     
## -- Conflicts ------------------------------------------ tidyverse_conflicts() --
## x dplyr::filter() masks stats::filter()
## x dplyr::lag()    masks stats::lag()
## i Use the conflicted package (<http://conflicted.r-lib.org/>) to force all conflicts to become errors
## Rows: 318851 Columns: 26
## -- Column specification --------------------------------------------------------
## Delimiter: ","
## chr   (4): url, id, floor, constructionTime
## dbl  (21): Lng, Lat, Cid, DOM, followers, totalPrice, price, square, livingR...
## date  (1): tradeTime
## 
## i Use `spec()` to retrieve the full column specification for this data.
## i Specify the column types or set `show_col_types = FALSE` to quiet this message.
\end{verbatim}

\begin{verbatim}
##    totalPrice        Lat             DOM           followers     
##  Min.   : 163   Min.   :39.63   Min.   :  1.00   Min.   :  0.00  
##  1st Qu.: 640   1st Qu.:39.90   1st Qu.: 18.00   1st Qu.:  8.00  
##  Median : 890   Median :39.96   Median : 47.00   Median : 26.00  
##  Mean   :1003   Mean   :39.97   Mean   : 69.14   Mean   : 41.49  
##  3rd Qu.:1230   3rd Qu.:40.05   3rd Qu.: 96.00   3rd Qu.: 54.00  
##  Max.   :4900   Max.   :40.24   Max.   :659.00   Max.   :633.00  
##      price            square          floor       renovationCondition
##  Min.   :  6272   Min.   :130.0   Min.   : 3.00   Min.   :1.000      
##  1st Qu.: 42184   1st Qu.:138.4   1st Qu.: 8.00   1st Qu.:3.000      
##  Median : 58103   Median :149.7   Median :15.00   Median :4.000      
##  Mean   : 61234   Mean   :163.1   Mean   :15.77   Mean   :3.352      
##  3rd Qu.: 76180   3rd Qu.:172.0   3rd Qu.:23.00   3rd Qu.:4.000      
##  Max.   :149551   Max.   :532.2   Max.   :63.00   Max.   :4.000      
##      subway       constructionTime communityAverage
##  Min.   :0.0000   Min.   :1965     Min.   : 24941  
##  1st Qu.:0.0000   1st Qu.:2001     1st Qu.: 43307  
##  Median :1.0000   Median :2004     Median : 58188  
##  Mean   :0.5595   Mean   :2004     Mean   : 61240  
##  3rd Qu.:1.0000   3rd Qu.:2008     3rd Qu.: 75490  
##  Max.   :1.0000   Max.   :2016     Max.   :152118
\end{verbatim}

Step2 Set up Model

\begin{Shaded}
\begin{Highlighting}[]
\NormalTok{model }\OtherTok{\textless{}{-}} \FunctionTok{lm}\NormalTok{(df}\SpecialCharTok{$}\NormalTok{price }\SpecialCharTok{\textasciitilde{}}\NormalTok{ df}\SpecialCharTok{$}\NormalTok{Lat }\SpecialCharTok{+}\NormalTok{ df}\SpecialCharTok{$}\NormalTok{DOM }\SpecialCharTok{+}\NormalTok{ df}\SpecialCharTok{$}\NormalTok{followers }\SpecialCharTok{+}\NormalTok{ df}\SpecialCharTok{$}\NormalTok{communityAverage }\SpecialCharTok{+}\NormalTok{ df}\SpecialCharTok{$}\NormalTok{square }\SpecialCharTok{+}\NormalTok{ df}\SpecialCharTok{$}\NormalTok{floor }\SpecialCharTok{+}\NormalTok{ df}\SpecialCharTok{$}\NormalTok{renovationCondition }\SpecialCharTok{+}\NormalTok{ df}\SpecialCharTok{$}\NormalTok{subway }\SpecialCharTok{+}\NormalTok{ df}\SpecialCharTok{$}\NormalTok{constructionTime }\SpecialCharTok{+}\NormalTok{ df}\SpecialCharTok{$}\NormalTok{followers}\SpecialCharTok{:}\NormalTok{df}\SpecialCharTok{$}\NormalTok{renovationCondition)}
\NormalTok{model}
\end{Highlighting}
\end{Shaded}

\begin{verbatim}
## 
## Call:
## lm(formula = df$price ~ df$Lat + df$DOM + df$followers + df$communityAverage + 
##     df$square + df$floor + df$renovationCondition + df$subway + 
##     df$constructionTime + df$followers:df$renovationCondition)
## 
## Coefficients:
##                         (Intercept)                               df$Lat  
##                          -1.600e+05                           -8.337e+03  
##                              df$DOM                         df$followers  
##                          -6.608e+00                           -4.164e+01  
##                 df$communityAverage                            df$square  
##                           1.023e+00                           -5.919e+01  
##                            df$floor               df$renovationCondition  
##                           7.371e-01                            5.933e+02  
##                           df$subway                  df$constructionTime  
##                           1.220e+02                            2.499e+02  
## df$followers:df$renovationCondition  
##                           3.812e+00
\end{verbatim}

Step 2 do some data exploration using plots to see any patterns

\begin{Shaded}
\begin{Highlighting}[]
\CommentTok{\#data exploration}
\FunctionTok{library}\NormalTok{(ggplot2)}

\CommentTok{\# 1.interaction term exploration for followers:renovationCondition}
\CommentTok{\# different predictors may have interaction on each other, draw this interaction plot to see any potential interaction, if the slopes of lines are obviously different then try to consider an interaction term in your model.}
\CommentTok{\# you may want to set a baseline/reference level for your interaction term.}
\NormalTok{p }\OtherTok{\textless{}{-}} \FunctionTok{ggplot}\NormalTok{(df, }\FunctionTok{aes}\NormalTok{(}\AttributeTok{x=}\NormalTok{df}\SpecialCharTok{$}\NormalTok{followers, }\AttributeTok{y=}\NormalTok{df}\SpecialCharTok{$}\NormalTok{price, }\AttributeTok{color=}\NormalTok{df}\SpecialCharTok{$}\NormalTok{renovationCondition, }\AttributeTok{group =}\NormalTok{ df}\SpecialCharTok{$}\NormalTok{renovationCondition)) }\SpecialCharTok{+}  \CommentTok{\# Set variables and differentiate lines by factor \textquotesingle{}renovationCondition\textquotesingle{}}
  \FunctionTok{geom\_point}\NormalTok{() }\SpecialCharTok{+}  \CommentTok{\# Add points}
  \FunctionTok{geom\_smooth}\NormalTok{(}\AttributeTok{method=}\StringTok{"lm"}\NormalTok{, }\AttributeTok{se=}\ConstantTok{FALSE}\NormalTok{) }\SpecialCharTok{+}  \CommentTok{\# Add lines for each group (no shading     for confidence interval)}
  \FunctionTok{labs}\NormalTok{(}\AttributeTok{title=}\StringTok{"Interaction Plot"}\NormalTok{, }\AttributeTok{x=}\StringTok{"followers"}\NormalTok{, }\AttributeTok{y=}\StringTok{"price per square"}\NormalTok{)}
\NormalTok{p}
\end{Highlighting}
\end{Shaded}

\begin{verbatim}
## Warning: Use of `df$followers` is discouraged.
## i Use `followers` instead.
\end{verbatim}

\begin{verbatim}
## Warning: Use of `df$price` is discouraged.
## i Use `price` instead.
\end{verbatim}

\begin{verbatim}
## Warning: Use of `df$renovationCondition` is discouraged.
## i Use `renovationCondition` instead.
## Use of `df$renovationCondition` is discouraged.
## i Use `renovationCondition` instead.
\end{verbatim}

\begin{verbatim}
## Warning: Use of `df$followers` is discouraged.
## i Use `followers` instead.
\end{verbatim}

\begin{verbatim}
## Warning: Use of `df$price` is discouraged.
## i Use `price` instead.
\end{verbatim}

\begin{verbatim}
## Warning: Use of `df$renovationCondition` is discouraged.
## i Use `renovationCondition` instead.
## Use of `df$renovationCondition` is discouraged.
## i Use `renovationCondition` instead.
\end{verbatim}

\begin{verbatim}
## `geom_smooth()` using formula = 'y ~ x'
\end{verbatim}

\includegraphics{STA302-Final-Project_files/figure-latex/unnamed-chunk-3-1.pdf}

\begin{Shaded}
\begin{Highlighting}[]
\CommentTok{\#interpretation:you notice that the interaction plot has category lines in diff slopes, so there is some impact of renovation condition on followers and thus affect price.}

\CommentTok{\# a histogram of response price}
\NormalTok{df }\SpecialCharTok{\%\textgreater{}\%} \FunctionTok{ggplot}\NormalTok{(}\FunctionTok{aes}\NormalTok{(}\AttributeTok{x =}\NormalTok{ price)) }\SpecialCharTok{+} 
  \FunctionTok{geom\_histogram}\NormalTok{(}\AttributeTok{bins =} \DecValTok{30}\NormalTok{, }\AttributeTok{fill =} \StringTok{\textquotesingle{}blue\textquotesingle{}}\NormalTok{, }\AttributeTok{color =} \StringTok{\textquotesingle{}black\textquotesingle{}}\NormalTok{) }\SpecialCharTok{+}
  \FunctionTok{theme\_minimal}\NormalTok{() }\SpecialCharTok{+}
  \FunctionTok{labs}\NormalTok{(}\AttributeTok{title =} \StringTok{"Histogram of price per square"}\NormalTok{, }\AttributeTok{x =} \StringTok{"price per square"}\NormalTok{, }\AttributeTok{y =} \StringTok{"Frequency"}\NormalTok{)}
\end{Highlighting}
\end{Shaded}

\includegraphics{STA302-Final-Project_files/figure-latex/unnamed-chunk-3-2.pdf}

\begin{Shaded}
\begin{Highlighting}[]
\CommentTok{\# 2.for interested predictors (examines linearity, constant variance by observing the feature such as skewness/spread...), uncomment them and replace with variables names.}

\CommentTok{\#draw a histogram of predictor}
\CommentTok{\#ggplot(data, aes(x = value)) + }
  \CommentTok{\#geom\_histogram(bins = 30, fill = \textquotesingle{}blue\textquotesingle{}, color = \textquotesingle{}black\textquotesingle{}) +}
  \CommentTok{\#theme\_minimal() +}
  \CommentTok{\#labs(title = "Histogram of \textquotesingle{}value\textquotesingle{}", x = "Value", y = "Frequency")}

\CommentTok{\# draw a boxplot of predictors}
\CommentTok{\#ggplot(data, aes(x = category, y = value, fill = category)) +}
  \CommentTok{\#geom\_boxplot() +}
  \CommentTok{\#theme\_minimal() +}
  \CommentTok{\#labs(title = "Boxplot of \textquotesingle{}value\textquotesingle{} by \textquotesingle{}category\textquotesingle{}", x = "Category", y = "Value")}

\CommentTok{\# draw a scatterplot of predictor vs response}
\CommentTok{\#ggplot(data, aes(x = xvalue, y = yvalue)) +}
  \CommentTok{\#geom\_point(aes(color = category)) + }
  \CommentTok{\#theme\_minimal() +}
  \CommentTok{\#labs(title = "Scatterplot of \textquotesingle{}yvalue\textquotesingle{} vs \textquotesingle{}xvalue\textquotesingle{}", x = "X Value", y = "Y Value")}
\end{Highlighting}
\end{Shaded}

Step 3 Assumptions Checking by residual plots

\begin{Shaded}
\begin{Highlighting}[]
\CommentTok{\#residual vs fitted}
\NormalTok{y\_hat }\OtherTok{\textless{}{-}} \FunctionTok{fitted}\NormalTok{(model)}
\NormalTok{e\_hat }\OtherTok{\textless{}{-}} \FunctionTok{resid}\NormalTok{(model)}
\FunctionTok{plot}\NormalTok{(}\AttributeTok{x =}\NormalTok{  y\_hat, }\AttributeTok{y =}\NormalTok{ e\_hat, }\AttributeTok{main=}\StringTok{"Figure1. Residual vs Fitted"}\NormalTok{, }\AttributeTok{xlab=}\StringTok{"Fitted values of price per square"}\NormalTok{,}\AttributeTok{ylab=}\StringTok{"Residuals"}\NormalTok{)}
\end{Highlighting}
\end{Shaded}

\includegraphics{STA302-Final-Project_files/figure-latex/unnamed-chunk-4-1.pdf}

\begin{Shaded}
\begin{Highlighting}[]
\CommentTok{\# residual vs construction year}
\FunctionTok{plot}\NormalTok{(}\AttributeTok{x =}\NormalTok{ df}\SpecialCharTok{$}\NormalTok{constructionTime, }\AttributeTok{y =}\NormalTok{ e\_hat, }\AttributeTok{main=}\StringTok{"Figure 2.Residual vs construction year"}\NormalTok{,}
     \AttributeTok{xlab=}\StringTok{"construction year"}\NormalTok{, }\AttributeTok{ylab=}\StringTok{"Residual"}\NormalTok{)}
\end{Highlighting}
\end{Shaded}

\includegraphics{STA302-Final-Project_files/figure-latex/unnamed-chunk-4-2.pdf}

\begin{Shaded}
\begin{Highlighting}[]
\CommentTok{\# residual vs other predictors}




\CommentTok{\# response vs fitted}
\FunctionTok{plot}\NormalTok{(}\AttributeTok{x =}\NormalTok{  y\_hat , }\AttributeTok{y =}\NormalTok{ df}\SpecialCharTok{$}\NormalTok{price   , }\AttributeTok{main=}\StringTok{""}\NormalTok{,}
     \AttributeTok{xlab=}\StringTok{"Fitted values"}\NormalTok{, }\AttributeTok{ylab=}\StringTok{"price per square"}\NormalTok{)}
\FunctionTok{abline}\NormalTok{(}\AttributeTok{a =} \DecValTok{0}\NormalTok{, }\AttributeTok{b =} \DecValTok{1}\NormalTok{, }\AttributeTok{lty=}\DecValTok{2}\NormalTok{)}
\end{Highlighting}
\end{Shaded}

\includegraphics{STA302-Final-Project_files/figure-latex/unnamed-chunk-4-3.pdf}

\begin{Shaded}
\begin{Highlighting}[]
\CommentTok{\# QQ plot}
\FunctionTok{qqnorm}\NormalTok{(e\_hat, }\AttributeTok{main =} \StringTok{"Normal Q{-}Q Plot"}\NormalTok{)}
\FunctionTok{qqline}\NormalTok{(e\_hat)}
\end{Highlighting}
\end{Shaded}

\includegraphics{STA302-Final-Project_files/figure-latex/unnamed-chunk-4-4.pdf}

\begin{Shaded}
\begin{Highlighting}[]
\CommentTok{\#pairwise scatterplots between predictors}
\FunctionTok{pairs}\NormalTok{(df[, }\FunctionTok{c}\NormalTok{(}\DecValTok{2}\SpecialCharTok{:}\DecValTok{11}\NormalTok{)], )}
\end{Highlighting}
\end{Shaded}

\includegraphics{STA302-Final-Project_files/figure-latex/unnamed-chunk-4-5.pdf}

Step 4 Box-cox Transformation

\begin{Shaded}
\begin{Highlighting}[]
\CommentTok{\# Model fix using box{-}cox power/log transformation}
\CommentTok{\# access the function}
\NormalTok{packageurl }\OtherTok{\textless{}{-}} \StringTok{"https://cran.r{-}project.org/src/contrib/Archive/pbkrtest/pbkrtest\_0.4{-}4.tar.gz"}
\FunctionTok{install.packages}\NormalTok{(packageurl, }\AttributeTok{repos=}\ConstantTok{NULL}\NormalTok{, }\AttributeTok{type=}\StringTok{"source"}\NormalTok{)}
\end{Highlighting}
\end{Shaded}

\begin{verbatim}
## Installing package into '/opt/r'
## (as 'lib' is unspecified)
\end{verbatim}

\begin{Shaded}
\begin{Highlighting}[]
\FunctionTok{install.packages}\NormalTok{(}\StringTok{"car"}\NormalTok{, }\AttributeTok{dependencies=}\ConstantTok{TRUE}\NormalTok{)}
\end{Highlighting}
\end{Shaded}

\begin{verbatim}
## Installing package into '/opt/r'
## (as 'lib' is unspecified)
\end{verbatim}

\begin{Shaded}
\begin{Highlighting}[]
\FunctionTok{library}\NormalTok{(car)}
\end{Highlighting}
\end{Shaded}

\begin{verbatim}
## Loading required package: carData
\end{verbatim}

\begin{verbatim}
## 
## Attaching package: 'car'
\end{verbatim}

\begin{verbatim}
## The following object is masked from 'package:dplyr':
## 
##     recode
\end{verbatim}

\begin{verbatim}
## The following object is masked from 'package:purrr':
## 
##     some
\end{verbatim}

\begin{Shaded}
\begin{Highlighting}[]
\DocumentationTok{\#\# Loading required package: carData}
\CommentTok{\# input the og\_model into boxCox}
\FunctionTok{boxCox}\NormalTok{(model)}
\end{Highlighting}
\end{Shaded}

\includegraphics{STA302-Final-Project_files/figure-latex/unnamed-chunk-5-1.pdf}

\begin{Shaded}
\begin{Highlighting}[]
\CommentTok{\# input the predictor columns}
\CommentTok{\# notice we can ONLY apply transformation on numerical variables \textgreater{} 0.}
\NormalTok{trans }\OtherTok{\textless{}{-}} \FunctionTok{powerTransform}\NormalTok{(}\FunctionTok{cbind}\NormalTok{(df[,}\FunctionTok{c}\NormalTok{(}\DecValTok{2}\NormalTok{,}\DecValTok{3}\NormalTok{,}\DecValTok{5}\NormalTok{,}\DecValTok{6}\NormalTok{,}\DecValTok{7}\NormalTok{,}\DecValTok{10}\NormalTok{,}\DecValTok{11}\NormalTok{)]))}
\FunctionTok{summary}\NormalTok{(trans)}
\end{Highlighting}
\end{Shaded}

\begin{verbatim}
## bcPower Transformations to Multinormality 
##                  Est Power Rounded Pwr Wald Lwr Bnd Wald Upr Bnd
## Lat                54.7781       54.78      43.5080      66.0482
## DOM                 0.2199        0.22       0.1959       0.2438
## price               0.4743        0.50       0.4314       0.5171
## square             -3.4499       -3.45      -3.6431      -3.2567
## floor               0.2863        0.33       0.2255       0.3472
## constructionTime   19.7198       19.72       9.9708      29.4688
## communityAverage    0.0956        0.10       0.0322       0.1591
## 
## Likelihood ratio test that transformation parameters are equal to 0
##  (all log transformations)
##                                        LRT df       pval
## LR test, lambda = (0 0 0 0 0 0 0) 2825.769  7 < 2.22e-16
## 
## Likelihood ratio test that no transformations are needed
##                                        LRT df       pval
## LR test, lambda = (1 1 1 1 1 1 1) 8308.974  7 < 2.22e-16
\end{verbatim}

\begin{Shaded}
\begin{Highlighting}[]
\CommentTok{\#now you may apply power transformation on these variables and fit a new transformed model, see Module4+5 Worksheet.}
\end{Highlighting}
\end{Shaded}

Step 5 Fit the new transformed model and RE-DO Step 3 to see any
assumptions violated.

Step 6 Deal with constant variance with Response price: Using the
variance stablizing transformation. Treat your newly fitted model as a
function of f(y),if you see any violation of constant variance such as
the fanning pattern in residual plots then transform \texttt{df\$price}
into \texttt{log(df\$price)} or \texttt{(df\$price)\^{}power} for the
model.

Step 7 If you have more time, please go to search some literatures about
your model. You may use different libraries/database.

最后,我们应该会在本周末和下周选一个时间一起想一下我们的flowchart.

\end{document}
